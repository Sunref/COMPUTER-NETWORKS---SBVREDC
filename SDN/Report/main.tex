\documentclass[
	% -- opções da classe memoir --
	12pt,				% tamanho da fonte
	openright,			% capítulos começam em pág ímpar (insere página vazia caso preciso)
	oneside,			% para impressão em recto e verso. Oposto a oneside
	a4paper,			% tamanho do papel. 
	% -- opções da classe abntex2 --
	chapter=TITLE,		% títulos de capítulos convertidos em letras maiúsculas
	%section=TITLE,		% títulos de seções convertidos em letras maiúsculas
	%subsection=TITLE,	% títulos de subseções convertidos em letras maiúsculas
	%subsubsection=TITLE,% títulos de subsubseções convertidos em letras maiúsculas
	% -- opções do pacote babel --
	english,			% idioma adicional para hifenização
	brazil,				% o último idioma é o principal do documento
	]{abntex2}


% ------------------------------------
% PACOTES
% ------------------------------------

% Pacotes fundamentais 
\usepackage{lmodern}			% Usa a fonte Latin Modern
\usepackage[T1]{fontenc}		% Selecao de codigos de fonte.
\usepackage[utf8]{inputenc}		% Codificacao do documento (conversão automática dos acentos)
\usepackage{indentfirst}		% Indenta o primeiro parágrafo de cada seção.
\usepackage{color}				% Controle das cores
\usepackage{graphicx}			% Inclusão de gráficos
\usepackage{microtype} 			% para melhorias de justificação
\usepackage{IFSULDEMINAS} 				% fomato específico relatório IFSULDEMINAS 
% Pacotes adicionais, usados no anexo do modelo de folha de identificação
\usepackage{multicol}
\usepackage{multirow}
	
% Pacotes adicionais, usados apenas no âmbito do Modelo Canônico do abnteX2
\usepackage{lipsum}				% para geração de dummy text

% Pacotes de citações
\usepackage[brazilian,hyperpageref]{backref}	 % Paginas com as citações na bibl
\usepackage[alf]{abntex2cite}	% Citações padrão ABNT

% ------------------------------------
% CONFIGURAÇÕES DE PACOTES
% ------------------------------------

% Configurações do pacote backref
% Usado sem a opção hyperpageref de backref
\renewcommand{\backrefpagesname}{Citado na(s) página(s):~}
% Texto padrão antes do número das páginas
\renewcommand{\backref}{}
% Define os textos da citação
\renewcommand*{\backrefalt}[4]{
	\ifcase #1 %
		Nenhuma citação no texto.%
	\or
		Citado na página #2.%
	\else
		Citado #1 vezes nas páginas #2.%
	\fi}%

% Informações de dados para CAPA
\titulo{Modelo de Relatório \\ Projeto Interdisciplinar III}
% ---------------------------------------------------
%Inserir nomes dos autores aqui
\autor{	AUTOR1 \\
		AUTOR2 \\
        AUTOR3}
% ---------------------------------------------------
\local{Poços de Caldas}
\data{2019}
\tipotrabalho{Relatório técnico}


% Configurações de aparência do PDF final

% alterando o aspecto da cor azul
\definecolor{blue}{RGB}{41,5,195}

% informações do PDF
\makeatletter
\hypersetup{
     	%pagebackref=true,
		pdftitle={\@title}, 
		pdfauthor={\@author},
    	pdfsubject={\imprimirpreambulo},
	    pdfcreator={LaTeX with abnTeX2},
		pdfkeywords={abnt}{latex}{abntex}{abntex2}{relatório técnico}, 
		colorlinks=true,       		% false: boxed links; true: colored links
    	linkcolor=blue,          	% color of internal links
    	citecolor=blue,        		% color of links to bibliography
    	filecolor=magenta,      		% color of file links
		urlcolor=blue,
		bookmarksdepth=4
}
\makeatother

% Espaçamentos entre linhas e parágrafos 

% O tamanho do parágrafo é dado por:
\setlength{\parindent}{1.3cm}

% Controle do espaçamento entre um parágrafo e outro:
\setlength{\parskip}{0.2cm}  % tente também \onelineskip

% compila o indice
\makeindex

% Início do documento
\begin{document}

% Seleciona o idioma do documento (conforme pacotes do babel)
%\selectlanguage{english}
\selectlanguage{brazil}

% Retira espaço extra obsoleto entre as frases.
\frenchspacing 

% ----------------------------------------------------------
% ELEMENTOS PRÉ-TEXTUAIS
% ----------------------------------------------------------

% Capa
\imprimircapa

% inserir o sumario
\pdfbookmark[0]{\contentsname}{toc}
\tableofcontents*
\cleardoublepage

\setlength{\absparsep}{18pt}
\begin{resumo}

    Este trabalho tem como finalidade explicar o que é uma \textit{Software Defined Networking} (SDN), seus conceitos e
    aplicações. O objetivo é destacar sua funcionalidade e vantagens de implementação em sistemas de telecomunicações,
    como tablets e smartphones.

    \textbf{Palavras-chave}: SDN; Redes; Software; Aplicação.
\end{resumo}

% ----------------------------------------------------------
% ELEMENTOS TEXTUAIS
% ----------------------------------------------------------
\textual

\chapter{Introdução}
\label{cap:intro}

SDN ou \textit{Software Defined Networking} (rede definida por software) é um modo de modelar a infraestrutura de uma rede que usa
controladores de \textit{software} para acesso de recursos, melhorar a eficiencia e o seu monitoramento. Diferentemente de uma rede tradicional que controlada por meio de \textit{hardware}.


%----------------------------------------------------------------------
\section{Motivação}
\label{sec:motivacao}

\lipsum[2]

%----------------------------------------------------------------------
\section{Objetivos}
\label{sec:objetivos}

\lipsum[2]

%----------------------------------------------------------------------
\section{Estrutura do Trabalho}
\label{sec:estrutura}

\lipsum[1]

\chapter{Referencial Teórico}
\label{cap:ref_teorico}

\lipsum[5]

\chapter{Pesquisa}
\label{cap:pesquisa}

\section{O que é a SND?}
Nos últimos anos com o crescimento de tablets, smartphones e outros dispositivos de transmissão multimídia
surgiu a necessidade de controle e operação de rede, essencial para suprir as demandas desses sistemas.
Portanto, nada mais é que uma arquitetura de redes entre computadores, visando gerenciar serviços de rede
utilizando de softwares em vez de dispositivos especializados para esse tipo de controle.

Sendo um sistema centralizado, é capaz de reservar ou preparar recursos para que a aplicação não tenha
obstáculos técnicos de hardware, possuíndo monitoramento inteligente que é feito para ser adaptável automaticamente
de acordo com o estado da rede, digitalizando a mesma.

A fim e lidar com melhores aplicações em nuvem, é capaz de automatizar, escalar e otimizar redes redes publicas e privadas
de serviços, além de banco de dados. Escalável com mudanças contínuas pelas quais serviços de operadoras e provedores de
internet não conseguem acompanhar \cite{stefanini_performance2025}.
Alguns exemplos de SDN incluem a OpenDaylight\footnote{Disponível em: https://www.opendaylight.org/},
ONUS\footnote{Disponível em: https://opennetworking.org/onos/}, Ryu\footnote{Disponível em: https://ryu-sdn.org/},
VMware NSX\footnote{Disponível em: https://www.vmware.com/}

\section{Como Funciona}
A SDN é formada por componentes que podem ou não estar estarem localizadas na mesma área física.
A fim de eliminar funções de de roteamento e encaminhamento de pacotes, a SDN implementa controladores
e os coloca acima de hardwares de rede na nuvem ou localmente, permitindo gerenciamento de rede
diretamente \cite{ibm_sdn}.

Os componentes que compõem uma SDN consistem em:
\begin{itemize}
	\item \textbf{Aplicações:} São as encarregadas por transmitir informações ou solicitações de disponibilidadealocação
	de rede. Sendo composta por dois tipos de interface de programação de aplicações (API), é notável citar
	\textit{Southbound} e \textit{Northbound}. Os controladores podem programar e configurar dispositivos de rede nessa
	API (\textit{Southbound}), recuperando informações sobre estados e topologia, recebendo notificações sobre congestionamento de pacotes
	e falhas de link. Já a \textit{Northbound} executa as mesmas funções que a API anterior, com diferença em viabilizar automatização de tarefas
	de gerenciamento de redes, facilitar a integração de sistemas em nuvem e outros tipos de aplicações.
	\item \textbf{Controladores:} Responsável por implementar funções de controle de redes e coordenar a comunicação com aplicações determinando
	o tráfego de pacotes de dados, os controladores oferecem uma perspectiva mais centralizada da rede, armazenando informações sobre o estado da
	mesma e toma decisões de como gerenciar dispositivos de rede conforme suas politicas.
	\item \textbf{Dispositivos de Rede:} São \textit{switches}, roteadores e pontos de acesso que fazem o fluxo de pacotes e recebem as instruções dos controladores e podem oferecer suporte a funcionalidades, como encaminhamento baseado em fluxo, qualidade de serviço e engenharia de tráfego. Nas SDN esses dispositivos podem ser simplificados e padronizados.
	\item \textbf{Sistema MANO\footnote{Management and Orchestration}:} MANO, ou gerenciamento de orquestração, interage com o controlador de SDN por meio da API \textit{Northbound}
	automatizando a utilização de recursos para rede e garantindo o autodesempenho e disponibilidade de serviços.
\end{itemize}
\cite{redhat_sdn}

\section{Tipos}
Existem quatro tipos de SDN que são considerados os principais. São eles:

\begin{itemize}
	\item \textbf{SDN aberta:} Os protocolos públicos são usados como controle de dispositivos tanto físicos quanto virtuais, e são
	responsáveis pelo roteamento dos pacotes de dados.
	\item \textbf{SDN de API:} Nesses casos, geralmente a \textit{Southbound} fica responsável pela organização e controle do fluxo para
	cada dispositivo.
	\item \textbf{Modelo de sobreposição:} Uma rede virtual acima do hardware físico oferecendo túneis que estabelecem canais de comunicação
	com centro de processamento de dados (CPD).
	\item \textbf{Modelo híbrido:} Combina as SDNs com redes tradicionais, atribuindo o protocolo certo para cada trafego. Frequentemente usada como complemento as SDNs originais.
\end{itemize}
\cite{ibm_sdn}

\section{Vantagens e Desvantagens}
As SDNs centralizam o controle e gerenciamento de rede, isso oferece vantagens que outras abordagens de rede não possuem.
Podemos citar:

\subsection{Agilidade e Flexibilidade}
Permite o balanceamento de fluxo de tráfego de acordo com a necessidade e do uso, reduzindo latência, aumentando a eficiência da rede.
As operadoras de rede também tem mais controle sobre a mesma, podendo alterar suas configurações, garantir recursos e expandir sua capacidade
\cite{ibm_sdn}.

\subsection{Redução de Custos}
Como as SDNs mantém sempre um tráfego contínuo, mesmo sendo um alto investimento a se fazer, gera ao departamento de TI (Tecnologia
da Informação) um alívio, reduzindo custos e melhorando a eficiência de serviços ao consumidor final \cite{stefanini_performance2025}.

\subsection{Controle}
Permite aos administradores que definam suas políticas a partir de um local central para controlar na rede os acessos e suas medidas
de seguranças. Sendo aplicável em nuvem publica, híbrida, privada e multinuvem \cite{ibm_sdn, stefanini_performance2025}.

\subsection{Simplicidade}
Podendo se basear em um único protocolo de comunicação com uma ambla variedade de dispositivos de hardware, oferecendo flexibilidade na escolha de
dispositivos de rede, gerando simplicidade \cite{ibm_sdn}.

\subsection{Modernização de Telecomuniçaões}
Combinado à maquinas virtuais e virtualização de redes, permite que as operadoras forneçam separação de rede e controle distinto aos clientes.
Auxilia os operadores a melhorar sua escalabilidade e fornecer largura de banda sob demanda ao clientes\cite{ibm_sdn}.

Porém, ainda é sucetível a erros e problemas, sendo o mais comum a ser citado o:

\subsection{Risco de Centralização}
Por ser um sistema centralizado, há um único potencial ponto de falha vulnerável, pois, como as SDNs induzem a criação de novos pontos de rede,
a mesma fica sucetível a vulnerabilidades e ataques cibernéticos. Algumas SDNs também são de código aberto, o que facilita a implementação de
código malicioso \cite{ibm_sdn, ufrj_sdn_2018}.

\chapter{Desenvolvimento}
\label{cap:desenvolvimento}

\lipsum[6]

\chapter{Conclusões}
\label{cap:conclusoes}

\lipsum[1]
Exemplo de citação indireta \cite{ableson2012android}




% ----------------------------------------------------------
% ELEMENTOS PÓS-TEXTUAIS
% ----------------------------------------------------------
\postextual

% ----------------------------------------------------------
% Referências bibliográficas
% ----------------------------------------------------------
\bibliography{referencias}


\end{document}
