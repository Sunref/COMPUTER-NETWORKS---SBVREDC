\chapter{Introdução}
\label{cap:intro}

\textit{Software Defined Networking} ou SDN (rede definida por software) é uma forma de modelar a infraestrutura de uma rede que usa
controladores de software para acesso de recursos, melhorar sua eficiencia e o seu monitoramento; diferentemente de uma rede
tradicional que é controlada por meio de hardware. Este trabalho tem como objetivos estudar seus conceitos, protocolos e
arquiteturas. Também seus principais usos, benefícios e desafios, além da apresentação de exemplos de aplicações.

\section{Objetivos}
\label{sec:objetivos}

Os objetivos deste documento é a apresentação do paradigma SDN, estuda-lo e apresenta-lo de forma sucinta. Apresentar
aplicações reais e seus benefícios e discutir seus desafios e limitações. Abaixo, segue a listagem dos objetivos específicos:

\begin{itemize}
    \item Estudar os conceitos, arquiteturas e protocolos do SDN;
    \item Apresentar aplicações reais do SDN;
    \item Discutir os benefícios do SDN;
    \item Discutir os desafios e limitações do SDN.
\end{itemize}